\section*{LỜI NÓI ĐẦU} % dấu * để không đánh sô thứ tự vào lời nói đầu
\thispagestyle{empty}

Cuộc cách mạng công nghiệp lần thứ tư đã và đang phát triển với tốc
độ rất nhanh, ảnh hưởng đến mọi mặt đời sống xã hội. Nội dung cốt lõi của cuộc cách mạng chính là sự kết hợp giữa 
khoa học công nghệ, trí tuệ nhân tạo và sự sáng tạo của con người. Đối với Việt Nam đang trong quá trình công nghiệp hoá, 
hiện đại hoá, việc áp dụng được những công nghệ mới trong một số lĩnh vực thiết yếu của xã hội, đặc biệt trong ngành y tế, 
chính là nền tảng quan trọng để chăm sóc sức khoẻ con người, từ đó tạo nên những con người với sức khoẻ tốt nhất, sẵn sàng 
đóng góp cho sự phát triển của đất nước.

Hiểu và nhận thức sâu sắc được điều đó, chúng em thực hiện đồ án với đề tài "Xây dựng hệ thống theo dõi và quản lý dữ liệu điện tim",
với mong muốn có thể áp dụng những kiến thức mình học được, kết nối các thiết bị IOT chăm sóc sức khoẻ đến với người dùng một cách
hiệu quả nhất, đồng thời thiết kế ra một ứng dụng giúp các bạn phát triển phần cứng, firmware có thể kiểm thử thiết bị
một cách dễ dàng hơn.

Trong quá trình thực hiện đồ án, chúng em đã nhận được sự hướng dẫn và hỗ trợ rất nhiệt tình của các thầy cô và các anh/chị/bạn trong
các lab trực thuộc trường Điện - Điện tử. Đầu tiên, chúng em xin chân thành cảm ơn TS. Nguyễn Thị Kim Thoa và TS. Hàn Huy Dũng đã
trực tiếp hướng dẫn và chỉ ra những điểm cần khắc phục trong quy trình làm đồ án cũng như thiết kế hệ thống của chúng em. Ngoài ra,
chúng em cảm thấy rất may mắn khi trong đồ án lần này được kết hợp và làm việc với các anh/chị/bạn ở nhóm firmware SPARC Lab và
anh/chị/bạn ở BKIC Lab, chúng em đã học hỏi được rất nhiều khi được làm việc cùng các thành viên của hai Lab.

Dù đã kiểm tra nhiều lần tuy nhiên đồ án của chúng em chắc chắn không thể tránh khỏi những thiếu sót và hạn chế, chúng em rất mong nhận được sự đóng góp ý
kiến của các thầy cô, bạn đọc để hoàn thiện và phát triển đề tài hơn nữa.

Chúng em xin chân thành cảm ơn!



\cleardoublepage
